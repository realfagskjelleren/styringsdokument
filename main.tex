\documentclass{article}
\usepackage{graphicx}

\title{Styringsdokumenter - Realfagskjelleren}
\date{Sist revidert: 25.03.35}

\begin{document}

\maketitle

\section{Leder}
\subsection{Generelt}
Foreningens leder har det overordnede ansvaret for foreningens helhetlige drift og virksomhet, og at denne er i samsvar med vedtektene. Leder innehar også rollen som foreningens talsperson i mediesaker og liknende.

\subsection{Styret}
Styreleder plikter i semesterm ̊anedene  ̊a avholde minst ett styremøte hvor leder fungerer som møteleder. I tillegg til dette, skal leder se til at generalforsamlingen gjennomføres. Leder skal arbeide for et godt samhold innad i styret, samt være styrets kontaktperson utad. Leder skal også tilse at styrets øvrige medlemmer forholder seg til både vedtektene og sitt styringsdokument.

\subsection{Annet}
Leder skal føre foreningens medlemsliste, og holde  ̊arlige medlems-samtaler for  ̊a sørge for tilstrekkelig inkludering av medlemmene, og for  ̊a hindre frafall. Leder skal ogs ̊a kalle inn til medlemsmøter s ̊a ofte styrets leder eller et flertall av styrets medlemmer finner det nødvendig.

\subsection{Juridisk}
Leder er styrer for Realfagskjelleren som bevillingshaver i Trondheim kommune. Leder plikter seg til å oppdatere stedfortreder (kasserer/daglig leder), og sørge for at Realfagskjelleren følger de gitte lover og instrukser gitt fra alkoholloven, men også kommunen. Dette innebærer en kunnskapsprøve i alkoholloven, samt etablererprøven og kurs i ansvarlig vertskap. Til slutt må leder oppfylle de krav som alkoholloven stiller til leder som person (jf. alkoholloven \S 1-7b og \S 1-7c).

\section{Nestleder}
\subsection{Generelt}
Foreningens nestleder er leders stedfortreder i styret. Nestleder administrerer varelageret for interne profilartikler, og bestiller ved behov. I tilegg fyller nestleder oppgaver når de oppstår, eller blir delegert av leder.

\subsection{Drift}
Nestleder vil være leders stedfortreder ved midlertidig eller permanent frafall. Ved midlertidig frafall skal nestleder følge de instrukser som leder av foreningen har gitt. Ved permanent frafall inntrer nestleder som leder og skal ogs ̊a følge leders styringsdokument. Ny nestleder velges i henhold til vedtektene.

\section{Kasserer}
\subsection{Generelt}
Kasserer har det overordnede ansvaret for foreningens økonomi og regnskap, samt annet administrativt arbeid spesifisert under. Foreningens kasserer er foreningens daglige leder, men benytter hovedsakelig tittelen kasserer for ikke  ̊a forveksles med leder. Styringsdokumentet til kasserer er å anse som en stillingsinstruks for daglig leder.

\subsection{Styret}
Kasserer skal informere styret om hendelser som er av styrets interesse.

\subsection{Administrasjon}
Kasserer fører foreningens regnskap og godkjenner dens inntekter og utgifter. Kasserer er ansvarlig for at foreningen økonomisk driftes i henhold til vedtekter og formål. Utover dette skal kasserer administrere foreningens bankkonti, betalingsplattformer og registre. Dette inkluderer ogs ̊a merverdiavgifts-meldinger.

\subsection{Juridisk}
Kasserer er daglig leder og skal sørge for at foreningens regnskap er i samsvar med norske lover og forskrifter, samt at formuesforvaltningen er ordnet på en betryggende måte. Kasserer må også være kjent med bokføringsloven og merverdiavgiftsloven.\newline
Kasserer er stedfortreder for Realfagskjelleren som bevillingshaver i Trondheim kommune. Kasserer plikter seg til å følge opp styrer (leder), og sørge for at Realfagskjelleren følger de gitte lover og instrukser gitt fra alkoholloven, men også kommunen. Dette innebærer en kunnskapsprøve i alkoholloven, samt etablererprøven og kurs i ansvarlig vertskap. Til slutt må kasserer oppfylle de krav som alkoholloven stiller til kasserer som person (jf. alkoholloven \S 1-7b og \S 1-7c).

\section{Kjellerkontakter}
\subsection{Generelt}
Linjeforeningenes kjellerkontakter skal virke som bindeledd mellom linjeforeningene og Realfagskjelleren, både innad og utad. \newline
Kjellerkontakter har hovedansvaret for ̊a finne vakter til arrangementer booket av sin respektive linjeforening. Videre har kjellerkontakter ansvar for å drive promo for Realfagskjelleren på linjeforeningenes interne kanaler. En kjellerkontakt skal også aktivt ta del i opptaksprosessen for sin linjeforening.

\section{Innkjøpsansvarlig}
\subsection{Generelt}
Innkjøpsansvarlig er ansvarlig for kjellerens driftsrelaterte varebeholdning. \newline
Videre har rollen ansvar for å organisere mottak av vareleveringer, pantehenting og innkjøpsrunder. Rollen er ikke ansvarlig for å gjennomføre disse selv, men skal delegere oppgavene og sikre at de blir gjennomført. \newline
Innkjøpsansvarlig er også ansvarlig for Realfagskjellerens bedriftsavtaler med Vinmonopolet, Ringnes og andre bedrifter. \newline
Innkjøpsansvarlig plikter å holde jevn kontakt med kasserer vedrørende innkjøp, returer og andre økonomirelaterte hendelser. Sammen med kasserer skal innkjøpsansvarlig beregne kalkyler for Realfagskjellerens produkter minst en gang hvert semester og deretter presentere disse for styret. Innkjøpsansvarlig skal jevnlig vurdere hvilke produkter som skal selges ved Realfagskjelleren. 

\section{SoMe-Ansvarlig}
\subsection{Generelt}
SoMe-ansvarligs ansvar er Realfagskjellerens profilering utad. SoMe-ansvarlig har ansvar for Realfagskjellerens grafiske design som inkluderer, men ikke begrenses til, bannere for arrangementer, maler for menyer og design av profileringsartikler for medlemmer i samarbeid med nestleder. \newline
Videre skal SoMe-ansvarlig forsikre at Realfagskjellerens offentlige arrangementer informeres om via Realfagskjellerens kanaler, f.eks. ved å legge ut arrangement på Facebook i god tid før arrangementet. \newline
SoMe-ansvarlig plikter å sette seg inn i Alkohollovens reglementer vedrørende markedsføring og etterkomme disse.

\end{document}
